\documentclass[11pt, a4paper, twoside]{article} % , draft
\usepackage[utf8]{inputenc}

\usepackage{enumitem} % customiçe item dots etc
\usepackage{textgreek} % obv
\usepackage{physics} % for easy derivative notation
\usepackage{amsmath}
\usepackage{amsthm} %theorems
\usepackage{amssymb}
\usepackage{mathtools} % for matrices with blocks inside
\usepackage[scr=boondoxo]{mathalfa}
\usepackage{pst-node}%
\usepackage{mathrsfs}
\DeclareMathAlphabet{\mathpzc}{OT1}{pzc}{m}{it}

\newcommand{\mc}{\multicolumn{1}{c}}
\newcommand{\R}{\mathbb{R}} % command for real R
\newcommand{\Holo}{\mathcal{H}}
\newcommand{\M}{\mathcal{M}}
\newcommand{\C}{\mathbb{C}}
\newcommand{\N}{\mathbb{N}}
\newcommand{\z}{\mathpzc{s}}
\newcommand{\p}{\mathpzc{r}}
\newcommand{\s}{\mathbb{S}}
\newcommand{\W}{\mathbb{W}}
\newcommand{\U}{\mathscr{U}}
\newcommand{\Lg}{\mathscr{L}}
\newcommand{\x}{\mathcal{X}}

\usepackage{csquotes}
\MakeOuterQuote{"}
\setlength{\parskip}{0.3 cm}

\usepackage{fancyhdr}

%\usepackage{nath} % authomatic parenthesis stuff
%\delimgrowth=1
\usepackage[left=2cm, right=2cm, top=2.1cm, bottom=2.1cm]{geometry} % set custom margins
\usepackage{graphicx} % to insert figures
\usepackage{grffile}
\graphicspath{{Figures/}} % define the figure folder path
\usepackage{subcaption} % for multiple figures at once each with a caption
\usepackage{multirow} %multirow in tables

\usepackage{caption}
\captionsetup[figure]{font=footnotesize} %adjust caption size
\captionsetup[table]{font=footnotesize} %adjust caption size

\usepackage{booktabs} % for pretty tabs in tables
\usepackage{siunitx} % Required for alignment
\captionsetup{labelfont=bf} % bold face captations

\usepackage{hyperref} % makes every reference a hyperlink
\hypersetup{
    colorlinks=true,
    linkcolor=violet,
    filecolor=[rgb]{0.69, 0.19, 0.38},      
    urlcolor=[rgb]{0.0, 0.81, 0.82},
    citecolor=[rgb]{0.69, 0.19, 0.38}
}

\usepackage{epigraph} % for quotations in teh begginig
\setlength\epigraphwidth{8cm}
\setlength\epigraphrule{0pt}
\usepackage{etoolbox}
\makeatletter
\patchcmd{\epigraph}{\@epitext{#1}}{\itshape\@epitext{#1}}{}{}
\renewcommand{\qedsymbol}{o.\textepsilon.\textdelta}

\newtheorem{prop}{Proposition} %so I can use propositions
\newtheorem{cor}{Corollary} %so I can use corollaries
\newtheorem{defi}{Definition} %so I can use corollaries

\makeatother % all this is for the epigraph
\usepackage{tocloft}

\usepackage{imakeidx} % make index

\makeindex[columns=3, title=Alphabetical Index, intoc]

%\title{\vspace{-2.5cm} {\bf Can we make the Exponential scaling in Time\\ be Linear in Time if Parallelized Exponentially? \\ {\em - Part 2 -}} \vspace{-0.4cm}  }
\title{\vspace{-2cm} {\bf End-to-End Creation of a \\Conical Refraction Polarimeter}\\{\small by {\em Xabier Oyanguren Asua}}\vspace{-0.8cm}}
\date{\vspace{-11ex}}
\let\clipbox\relax
\usepackage{adjustbox}
\newcolumntype{?}{!{\vrule width 1.5pt}}
\usepackage{abstract}
\setlength{\absleftindent}{0mm}
\setlength{\absrightindent}{0mm}

\usepackage{tcolorbox}
\DeclareRobustCommand{\mybox}[2][gray!10]{%
\begin{tcolorbox}[   %% Adjust the following parameters at will.
        left=0.2cm,
        right=0.2cm,
        top=0.15cm,
        bottom=0.15cm,
        colback=#1,
        colframe=#1,
        width=\dimexpr\textwidth\relax, 
        enlarge left by=0mm,
        boxsep=5pt,
        arc=0pt,outer arc=0pt,
        ]
        #2
\end{tcolorbox}
}


\usepackage{anyfontsize}

\NewDocumentEnvironment{kapituloBerria}{mm}
{\clearpage           % we want a new page          %% I commented this
   \thispagestyle{empty}% no header and footer
   \vspace*{\stretch{2}}% some space at the top
   \raggedleft          % flush to the right margin
   {\textbf{{\fontsize{60}{40}\selectfont \hspace{+9.5cm}#1 \newline \newline}}}
   \bf
   \fontsize{30}{20}\selectfont
  }
  {\par % end the paragraph
   \vspace{\stretch{3}} % space at bottom is three times that at the top
   \normalfont
      \fontsize{15}{20}\selectfont
      \vspace{-1cm}
   \begin{flushleft}{ \textit{#2}  }
   \end{flushleft}
   \clearpage           % finish off the page
  }
  
%\newenvironment{kapituloBerria}[2]
  

\usepackage{listings}
\usepackage{xcolor}
\lstset{language=C++,
                basicstyle=\ttfamily,
                keywordstyle=\color{blue}\ttfamily,
                stringstyle=\color{red}\ttfamily,
                commentstyle=\color{green}\ttfamily,
                morecomment=[l][\color{magenta}]{\#}
    backgroundcolor=\color{black!5}, % set backgroundcolor
    basicstyle=\footnotesize,% basic font setting
}

\begin{document}

\clearpage
%% temporary titles
% command to provide stretchy vertical space in proportion
\newcommand\nbvspace[1][3]{\vspace*{\stretch{#1}}}
% allow some slack to avoid under/overfull boxes
\newcommand\nbstretchyspace{\spaceskip0.5em plus 0.25em minus 0.25em}
% To improve spacing on titlepages
\newcommand{\nbtitlestretch}{\spaceskip0.6em}
\pagestyle{empty}
\begin{center}
\bfseries
\nbvspace[1]
\Huge
{\nbtitlestretch
{\normalsize END-TO-END CREATION OF A }\\
\Huge CONICAL REFRACTION \\
POLARIMETER }

\nbvspace[4]
{\small

BUILDING A STATE OF THE ART\vspace{0.2cm}\\ AFFORDABLE DEVICE\vspace{0.2cm}\\ TO MONITOR LINEAR POLARIZATION\\
}

\normalsize

\nbvspace[5]
\small BY\\
\Large Xabier Oyanguren Asua\\[0.5em]
%\footnotesize Can we really foresee the future of the Universe?"\\

\nbvspace[5]

\small THESIS SUPERVISOR \vspace{0.15cm}\\
\Large Àlex Turpin Avilés

\nbvspace[1]

\includegraphics[width=2.5in]{UAB.png}
\normalsize
\vspace{-0.5cm}
%\small Thesis Directors: \\
%\nbvspace[0.2]
%\large Jordi Mompart Penina \\
%Xavier Oriols Pladevall\\
%\nbvspace[1]

%Universitat Autònoma de Barcelona\\
\large
%DEGREE FINAL DISSERTATION \\
\small
Bachelor's Thesis \\ Computational Mathematics and Data Analytics \\
\vspace{0.1cm}
\small
2021-2022
\nbvspace[1]
\end{center}
\newpage
\null
\clearpage

\maketitle
\vspace{-0.3cm}
\pagenumbering{gobble}
\setlength{\cftbeforesecskip}{0.4cm}
\setlength{\cftbeforesubsecskip}{0.4cm}
\setlength{\cftbeforesubsubsecskip}{0.25cm}

\tableofcontents
\clearpage
\pagenumbering{arabic}
\setcounter{page}{1}
\vspace{-0.3 cm}

%\pagestyle{empty}
\pagestyle{fancy}

\section*{\centering \huge{Introduction}\vspace{-0.3cm}}
\noindent\rule{\textwidth}{0.4pt}
\addcontentsline{toc}{section}{Introduction}
\section*{Abstract}


\section*{Objectives}\vspace{-0.2cm}



\section*{Guideline}\vspace{-0.2cm}


\newpage

%%%%%%%%%%%%%%%%%%%%%%%%%%%%%%%%%%%%%%%%%%%%%%%%%%%%%%%%%%%%%%%%%%%%%%%%%%%%%%%%%%%%%%%%%%%%%%%%%%%%%%%%%%%%%%%%%%%%%%%%%%%%%%%%%%%%%%%%%%%%%%%%%%%%%%%%%%%%%%%%%%%%%%%%%%%%%
\newpage


%\begin{kapituloBerria}{Part A}{ Understanding the Underlying Phenomenon}
%Conical Refraction Essentials 
%\end{kapituloBerria}
%\newpage
%\fancyhead[L]{\null}
%\fancyhead[R]{\null}
%\null
%\clearpage


\section*{\centering \huge{Part A: Conical Refraction Essentials}\vspace{-0.3cm}}
\noindent\rule{\textwidth}{0.4pt}
\vspace{-0.2cm}
\addcontentsline{toc}{section}{Part A: Conical Refraction Essentials}

In this first section, we will review the basic theoretical explanation of the Conical Refraction (CR) phenomenon in which the developed device will be based.
 
\section*{A.1. The Revival of Conical Refraction }
\addcontentsline{toc}{subsection}{1. The Revival of Conical Refraction }
Sartun Hamiltonegaz etc, esan zer dan en sí, ta zer implikeu ban beren egunien, zelan gero aztute geratu zan hasta tal eta formulaziño modernoak tal. Ipiñi dibujo klasiko bat del fenómeno, el cono ese, nik einde eskuz.

\section*{A.2. Berry's Mathematical Description of the Phenomenon }
\addcontentsline{toc}{subsection}{2. Berry's Mathematical Description of the Phenomenon }
En más o menos detalle guredozulez azaldu fenomenoan matematikie eta batezbe heldu formula finalatara

\section*{A.3. Simulating the Phenomenon }
\addcontentsline{toc}{subsection}{3. Simulating the Phenomenon }
Azaldu GPU/CPU tradeoffa, zelan implemente doten eta jarri imagenak. Azaldu zelan si lienar pol tal, si cricular pol tal imagenakaz. Sugeridu zer alko genun ein orduen linear polrztion aldaketak antzemateko.

\section*{A.4. A Natural Polarimeter}
\addcontentsline{toc}{subsection}{4. A Natural Polarimeter }
Esan zelan alko zendun argixen polarizaziño tal danak deskribatu einde bi besogaz et al, baia zelan simplifike al dan ze kiralidade temak LPgaz nahiko eta hori da polarimetroan merkatu handixetako bat.

\section*{Apppendix $\alpha$: Jones and Stokes Representations of Polarization}
\addcontentsline{toc}{subsection}{Apppendix $\alpha$: Jones and Stokes Representations of Polarization }



\newpage

\fancyhead[OL]{\em Part A: Conical Refraction Essentials}
\fancyhead[OR]{ Section title}

\fancyhead[EL]{ Section title}
\fancyhead[ER]{\em Bachelor's Thesis: Conical Refraction Polarimeter}
Ba hori



%%%%%%%%%%%%%%%%%%%%%%%%%%%%%%%%%%%%%%%%%%%%%%%%%%%%%%%%%%%%%%%%%%%%%%%%%%%%%%%%%%%%%%%%%%%%%%%%%%%%%%%%%%%%%%%%%%%%%%%%%%%%%%%%%%%%%%%%%%%%%%%%%%%%%%%%%%%%%%%%%%%%%%%%%%%%%
\newpage


%\begin{kapituloBerria}{Part B}{ The Path Towards a Fully Functional Portable Prototype}
%Designing the Hardware
%\end{kapituloBerria}
%\newpage
%\fancyhead[L]{\null}
%\fancyhead[R]{\null}
%\null
%\clearpage


\section*{\centering \huge{Part B: Designing the Hardware}\vspace{-0.3cm}}
\noindent\rule{\textwidth}{0.4pt}

\addcontentsline{toc}{section}{Part B: Designing the Hardware}

\section*{B.1. Experimental Implementations of the Polarimeter}
\addcontentsline{toc}{subsection}{1. Experimental Implementations of the Polarimeter}
Azaldu bakjoitza dibujo bategaz eta identifike problemak, edo gubazun hori hobeto alrgazki batzuk ipiñitze hurrengo seksiñoan, esaten posibles razones a las ke se deba, al dozu citeu ah artikuloa ta beran imagena bebai.


\section*{B.2. Main Non-Idealities in the Experimental Rings}
\addcontentsline{toc}{subsection}{2. Main Non-Idealities in the Experimental Rings}

\section*{B.3. Final Proofs of Concept}
\addcontentsline{toc}{subsection}{3. Proofs of Concept}
Bat ya einde egioin zana, en el ke implementé la GUI akella, argazki bat, con fotos de la GUI etc.

Bestie, lortzen badogun eitzie housiñegaz y tal, planoa ein eskuz ni ke sea

COST ANALYSIS EGIN

\section*{Appendix $\beta$: Employed Optical Elements}
\addcontentsline{toc}{subsection}{Appendix $\beta$: Employed Optical Elements}



\newpage


\fancyhead[OL]{\em Part B: Designing the Hardware}
\fancyhead[OR]{ Section title}

\fancyhead[EL]{\em Section title}
\fancyhead[ER]{\em Bachelor's Thesis: Conical Refraction Polarimeter}
Ba hori



%%%%%%%%%%%%%%%%%%%%%%%%%%%%%%%%%%%%%%%%%%%%%%%%%%%%%%%%%%%%%%%%%%%%%%%%%%%%%%%%%%%%%%%%%%%%%%%%%%%%%%%%%%%%%%%%%%%%%%%%%%%%%%%%%%%%%%%%%%%%%%%%%%%%%%%%%%%%%%%%%%%%%%%%%%%%%
\newpage


%\begin{kapituloBerria}{Part C}{ From Crescent Ring Images to Polarization Angle Differences }
%Designing the Software
%\end{kapituloBerria}
%\newpage
%\fancyhead[L]{\null}
%\fancyhead[R]{\null}
%\null
%\clearpage


\section*{\centering \huge{Part C: Designing the Software}\vspace{-0.3cm}}
\noindent\rule{\textwidth}{0.4pt}
\newpage

\addcontentsline{toc}{section}{Part C: Designing the Software}

Objetiboak argitu, ta komplikaziñoak 

\section*{C.1. Artificial Noise Generation}
\addcontentsline{toc}{subsection}{1. Artificial Noise Generation}

\section*{C.2. Simulated Image Datasets }
\addcontentsline{toc}{subsection}{2. Simulated Image Datasets}

\section*{C.2. Preprocessing }
\addcontentsline{toc}{subsection}{2. Preprocessing}

\section*{C.3. Embedding Space Algorithms}
\addcontentsline{toc}{subsection}{3. Embedding Space Algorithms}

\subsection*{C.3.1. Data Manifold Dimension Identification}
\addcontentsline{toc}{subsubsection}{3.1. Data Manifold Dimension Identification}

\subsection*{C.3.2. Metric Learning }
\addcontentsline{toc}{subsubsection}{3.2. Metric Learning}

\subsection*{C.3.3. Nearest Neighbors }
\addcontentsline{toc}{subsubsection}{3.3. Nearest Neighbors}

\section*{C.3. Geometric Algorithms}
\addcontentsline{toc}{subsection}{3. Geometric Algorithms}

Tos los geometricos y los optimizadores implementados.

\section*{C.4. Simulation Fitting Algorithms}
\addcontentsline{toc}{subsection}{4. Simulation Fitting Algorithms }

\section*{C.5. The Best Algorithm and Preprocessing}
\addcontentsline{toc}{subsection}{5. Looking for the Best Algorithm and Pre-processing }

\section*{Appendix $\gamma$: The Implemented Optimizers}
\addcontentsline{toc}{subsection}{Appendix $\gamma$: The Implemented Optimizers }

\fancyhead[OL]{\em Part C: Designing the Software}
\fancyhead[OR]{ Section title}

\fancyhead[EL]{\em Section title}
\fancyhead[ER]{\em Bachelor's Thesis: Conical Refraction Polarimeter}
Ba hori


%%%%%%%%%%%%%%%%%%%%%%%%%%%%%%%%%%%%%%%%%%%%%%%%%%%%%%%%%%%%%%%%%%%%%%%%%%%%%%%%%%%%%%%%%%%%%%%%%%%%%%%%%%%%%%%%%%%%%%%%%%%%%%%%%%%%%%%%%%%%%%%%%%%%%%%%%%%%%%%%%%%%%%%%%%%%%
\newpage

%
%\begin{kapituloBerria}{Part D}{ Performance Analysis, Commercial Polarimeters and Niche Outlining  }
%The Final Device
%\end{kapituloBerria}
%\newpage
%\fancyhead[L]{\null}
%\fancyhead[R]{\null}
%\null
%\clearpage


\section*{\centering \huge{Part D: The Final Device}\vspace{-0.3cm}}
\noindent\rule{\textwidth}{0.4pt}

\addcontentsline{toc}{section}{Part D: The Final Device}

\section*{D.1. Performance on Experimental Data }
\addcontentsline{toc}{subsection}{1. Performance on Experimental Data }

\section*{D.2. Commercial Polarimeters }
\addcontentsline{toc}{subsection}{2. Commercial Polarimeters }

\section*{D.3. Potential Niches }
\addcontentsline{toc}{subsection}{3. Potential Niches }

\newpage

\fancyhead[OL]{\em Part D: The Final Device}
\fancyhead[OR]{ Section title}

\fancyhead[EL]{\em Section title}
\fancyhead[ER]{\em Bachelor's Thesis: Conical Refraction Polarimeter}

\newpage
\fancyhead[OL]{}
\fancyhead[OR]{}
\fancyhead[EL]{}
\fancyhead[ER]{}

\section*{\centering \huge{Conclusions}\vspace{-0.3cm}}
\noindent\rule{\textwidth}{0.4pt}
\addcontentsline{toc}{section}{Conclusions}



\newpage
\begin{thebibliography}{1}
\addcontentsline{toc}{section}{References}


\bibitem{where}

\end{thebibliography}


\end{document}
